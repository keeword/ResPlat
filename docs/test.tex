\documentclass[a4paper,fancyhdr,fntef,hyperref]{ctexart}
\usepackage{geometry} 		% 設定邊界
% \geometry{
%   top=1in,
%   inner=1in,
%   outer=1in,
%   bottom=1in,
%   headheight=3ex,
%   headsep=2ex
% }
\geometry{
    top=2.5cm,          %设置页面上下左右各为25mm
    left=2.5cm,
    right=2.5cm,
    bottom=2.5cm,
    headheight=1.5cm,   %页眉所占高度
    footskip=1cm        %页脚所占高度
}
% \usepackage[T1]{fontenc}
% \usepackage{lmodern}
\usepackage{ifxetex,ifluatex}
\usepackage{ifthen,calc}
\usepackage{amssymb,amsmath}
\usepackage{txfonts}
% \usepackage{fixltx2e} % provides \textsubscript
% use upquote if available, for straight quotes in verbatim environments
% \IfFileExists{upquote.sty}{\usepackage{upquote}}{}
% use microtype if available
% \IfFileExists{microtype.sty}{\usepackage{microtype}}{}
% \usepackage[labelformat=simple]{subcaption}
\usepackage[neverdecrease]{paralist}
\usepackage{color}
\usepackage{fancyvrb}
\newcommand{\VerbBar}{|}
\newcommand{\VERB}{\Verb[commandchars=\\\{\}]}
\DefineVerbatimEnvironment{Highlighting}{Verbatim}{commandchars=\\\{\}}
% Add ',fontsize=\small' for more characters per line
\newenvironment{Shaded}{}{}
\newcommand{\KeywordTok}[1]{\textcolor[rgb]{0.00,0.44,0.13}{\textbf{{#1}}}}
\newcommand{\DataTypeTok}[1]{\textcolor[rgb]{0.56,0.13,0.00}{{#1}}}
\newcommand{\DecValTok}[1]{\textcolor[rgb]{0.25,0.63,0.44}{{#1}}}
\newcommand{\BaseNTok}[1]{\textcolor[rgb]{0.25,0.63,0.44}{{#1}}}
\newcommand{\FloatTok}[1]{\textcolor[rgb]{0.25,0.63,0.44}{{#1}}}
\newcommand{\CharTok}[1]{\textcolor[rgb]{0.25,0.44,0.63}{{#1}}}
\newcommand{\StringTok}[1]{\textcolor[rgb]{0.25,0.44,0.63}{{#1}}}
\newcommand{\CommentTok}[1]{\textcolor[rgb]{0.38,0.63,0.69}{\textit{{#1}}}}
\newcommand{\OtherTok}[1]{\textcolor[rgb]{0.00,0.44,0.13}{{#1}}}
\newcommand{\AlertTok}[1]{\textcolor[rgb]{1.00,0.00,0.00}{\textbf{{#1}}}}
\newcommand{\FunctionTok}[1]{\textcolor[rgb]{0.02,0.16,0.49}{{#1}}}
\newcommand{\RegionMarkerTok}[1]{{#1}}
\newcommand{\ErrorTok}[1]{\textcolor[rgb]{1.00,0.00,0.00}{\textbf{{#1}}}}
\newcommand{\NormalTok}[1]{{#1}}
\usepackage{longtable}
\usepackage{booktabs}
\usepackage{graphicx}
% We will generate all images so they have a width \maxwidth. This means
% that they will get their normal width if they fit onto the page, but
% are scaled down if they would overflow the margins.
\makeatletter
\def\maxwidth{\ifdim\Gin@nat@width>\linewidth\linewidth
\else\Gin@nat@width\fi}
\makeatother
\let\Oldincludegraphics\includegraphics
\renewcommand{\includegraphics}[1]{\Oldincludegraphics[width=\maxwidth]{#1}}
\ifxetex
  \usepackage[setpagesize=false, % page size defined by xetex
              unicode=false, % unicode breaks when used with xetex
              xetex]{hyperref}
\else
  \usepackage[unicode=true]{hyperref}
\fi
\hypersetup{breaklinks=true,
            bookmarks=true,
            pdfauthor={谢伟能},
            pdftitle={电信学院物资平台设计说明书},
            colorlinks=true,
            urlcolor=blue,
            linkcolor=magenta,
            pdfborder={0 0 0}}
\urlstyle{same}  % don't use monospace font for urls
\setlength{\parindent}{0pt}
%\setlength{\parskip}{6pt plus 2pt minus 1pt}
\setlength{\emergencystretch}{3em}  % prevent overfull lines

\usepackage{titling}
\setlength{\droptitle}{-8em} 	% 將標題移動至頁面的上面

% 页眉页脚
\usepackage{fancyhdr}
\usepackage{lastpage}
\pagestyle{fancyplain}

\CTEXoptions[today=big]
\CTEXsetup[name={第,节},number={\chinese{section}},format+={\bfseries},beforeskip={-10ex plus -.1ex minus -.1ex},afterskip={1ex plus .1ex minus .1ex}]{section}
\CTEXsetup[name={附件},number={\chinese{chapter}}]{appendix}
\CTEXsetup[name={第,篇},nameformat={\centering\bfseries},titleformat={\zihao{0}\bfseries}]{part}
\CTEXsetup[name={(,)},number={\arabic{paragraph}}]{paragraph}

\setcounter{secnumdepth}{3}

\title{电信学院物资平台设计说明书}
\author{谢伟能}
\date{\today}

\begin{document}
\maketitle
\newpage

\section{引言}\label{ux5f15ux8a00}

本文是电信学院物资平台的设计说明书。

按流程,应该是先有概要设计,再有详细设计,
但由于工期紧张,文档编写者缺乏经验等原因,
本设计说明书把两步合为一步,固称作设计说明书。

\subsection{编写目的}\label{ux7f16ux5199ux76eeux7684}

本文的对象是平台开发者,为他们提供一个详细的开发指引。

\subsection{项目背景}\label{ux9879ux76eeux80ccux666f}

本平台是为电信学院学生会更加方便管理物资而建设。

\subsection{定义}\label{ux5b9aux4e49}

平台:电信学院物资管理平台 本文、说明书:电信学院物资管理平台设计说明书
开发者:平台开发者

\subsection{参考资料}\label{ux53c2ux8003ux8d44ux6599}

电信学院物资平台 V1.0 产品需求说明书

\section{任务概述}\label{ux4efbux52a1ux6982ux8ff0}

\subsection{目标}\label{ux76eeux6807}

编码实现电信学院 V1.0 版本。

\subsection{运行环境}\label{ux8fd0ux884cux73afux5883}

LNMP------Linux,Nginx,MySQL,PHP

\subsection{需求概述}\label{ux9700ux6c42ux6982ux8ff0}

实现物资申请、物资信息修改等功能。

\subsection{条件和限制}\label{ux6761ux4ef6ux548cux9650ux5236}

在寒假(2015.01.16-2015.03.01)之内完成。

\section{总体设计}\label{ux603bux4f53ux8bbeux8ba1}

\subsection{功能架构}\label{ux529fux80fdux67b6ux6784}

\begin{figure}[htbp]
\centering
\includegraphics{./flow/function_struct.png}
\caption{功能架构}
\end{figure}

平台包括申请/审核系统和管理系统两部分,主要有:

\begin{itemize}
\itemsep1pt\parskip0pt\parsep0pt
\item
  用户登录/登出
\item
  用户管理
\item
  主页
\item
  物资申请/审核
\item
  工作室申请/审核
\item
  物资管理
\item
  工作室管理
\item
  用户通信
\end{itemize}

\subsection{系统架构}\label{ux7cfbux7edfux67b6ux6784}

本平台采用MVC的3层架构,主要包括:表现层、控制层、业务层。

表现层:主要负责用户交互和结果显示。
控制层:主要负责系统的控制访问、数据加载。
业务层:主要负责系统和数据库的交互。

用户 ←→ 表现层 ←→ 控制层 ←→ 业务层 ←→ 数据库

\subsection{模块设计}\label{ux6a21ux5757ux8bbeux8ba1}

\subsubsection{登录模块}\label{ux767bux5f55ux6a21ux5757}

登录模块主要用于用户登入和登出。

\paragraph{用户登入}\label{ux7528ux6237ux767bux5165}

\begin{figure}[htbp]
\centering
\includegraphics{./flow/login.png}
\caption{登入流程图}
\end{figure}

\paragraph{用户登出}\label{ux7528ux6237ux767bux51fa}

\begin{figure}[htbp]
\centering
\includegraphics{./flow/logout.png}
\caption{登出流程图}
\end{figure}

\subsubsection{用户模块}\label{ux7528ux6237ux6a21ux5757}

用户模块有添加帐号、更改密码、删除账号、修改用户信息等功能。

\paragraph{添加帐号}\label{ux6dfbux52a0ux5e10ux53f7}

只有管理员可以向系统添加帐号,不予注册。

\paragraph{更改密码}\label{ux66f4ux6539ux5bc6ux7801}

只有管理员可以更改密码,用户不可以。

\paragraph{删除帐号}\label{ux5220ux9664ux5e10ux53f7}

管理员可以删除其他帐号。

\paragraph{更改用户信息}\label{ux66f4ux6539ux7528ux6237ux4fe1ux606f}

管理员可以更改用户的权限、昵称等信息。

\subsubsection{主页模块}\label{ux4e3bux9875ux6a21ux5757}

主页模块是用户登录后的页面,包含

\begin{itemize}
\itemsep1pt\parskip0pt\parsep0pt
\item
  用户信息
\item
  申请/审核记录,包含两部分,自己个人的和所有人的,按时间排序,点击切换
\item
  物资列表
\end{itemize}

\subsubsection{物资申请/审核模块}\label{ux7269ux8d44ux7533ux8bf7ux5ba1ux6838ux6a21ux5757}

本模块有查看审核列表、查看详情、通过申请、拒绝申请等功能。

\paragraph{申请物资}\label{ux7533ux8bf7ux7269ux8d44}

所有用户都可以在此页面申请物资。

\paragraph{申请列表}\label{ux7533ux8bf7ux5217ux8868}

审核员可以在此页面查看所有的物资申请。

\paragraph{查看详情}\label{ux67e5ux770bux8be6ux60c5}

审核员可以在此页面查看某一物资申请的具体情况,
包括申请时间、申请人、申请原因、申请物品等等。

\paragraph{通过申请}\label{ux901aux8fc7ux7533ux8bf7}

通过一个或多个申请。

\paragraph{拒绝申请}\label{ux62d2ux7eddux7533ux8bf7}

拒绝一个或多个申请,并填写拒绝原因。
当批量拒绝时,所有都是同一拒绝原因。

\subsubsection{物资管理模块}\label{ux7269ux8d44ux7ba1ux7406ux6a21ux5757}

本模块有物资一览、添加物资、更新物资、删除物资、上传物资、添加分类、删除分类等功能。

\paragraph{物资一览}\label{ux7269ux8d44ux4e00ux89c8}

展示所有的物资情况。

\paragraph{添加物资}\label{ux6dfbux52a0ux7269ux8d44}

添加单个物资。

\paragraph{删除物资}\label{ux5220ux9664ux7269ux8d44}

删除单个或多个物资。

\paragraph{上传物资}\label{ux4e0aux4f20ux7269ux8d44}

上传excel文件,批量增加物资。

\paragraph{添加分类}\label{ux6dfbux52a0ux5206ux7c7b}

添加单个物资分类。

\paragraph{删除分类}\label{ux5220ux9664ux5206ux7c7b}

删除单个物资分类。

\subsubsection{工作室审核模块}\label{ux5de5ux4f5cux5ba4ux5ba1ux6838ux6a21ux5757}

本模块有查看申请列表、通过申请、拒绝申请等功能。

\paragraph{查看申请}\label{ux67e5ux770bux7533ux8bf7}

审核员可以在此页面查看所有未审核的物资申请。

\paragraph{通过申请}\label{ux901aux8fc7ux7533ux8bf7-1}

通过用户的工作室申请。

\paragraph{拒绝申请}\label{ux62d2ux7eddux7533ux8bf7-1}

拒绝用户的工作室申请。

\subsubsection{工作室管理模块}\label{ux5de5ux4f5cux5ba4ux7ba1ux7406ux6a21ux5757}

本模块暂无需求。

\subsubsection{用户通信模块}\label{ux7528ux6237ux901aux4fe1ux6a21ux5757}

本模块有发送通知、阅读通知、查看通知等功能。

\paragraph{发送通知}\label{ux53d1ux9001ux901aux77e5}

在审核员审核结束后,系统会自动发通知给用户。
管理员也可以发通知给任何用户。

\paragraph{查看通知}\label{ux67e5ux770bux901aux77e5}

用户在此页面查看所有发给自己的信息列表。

\paragraph{阅读通知}\label{ux9605ux8bfbux901aux77e5}

用户在此阅读某一通知的内容。

\subsection{数据库设计}\label{ux6570ux636eux5e93ux8bbeux8ba1}

\subsubsection{用户信息}\label{ux7528ux6237ux4fe1ux606f}

\paragraph{user表}\label{userux8868}

\begin{longtable}[c]{@{}llllll@{}}
\toprule\addlinespace
Field & Type & Null & Key & Default & Extra
\\\addlinespace
\midrule\endhead
id & int(10) unsigned & NO & PRI & NULL & auto\_increment
\\\addlinespace
username & varchar(255) & NO & UNI & NULL &
\\\addlinespace
nickname & varchar(255) & NO & & NULL &
\\\addlinespace
email & varchar(255) & NO & & NULL &
\\\addlinespace
phone & varchar(255) & NO & & NULL &
\\\addlinespace
password & varchar(255) & NO & & NULL &
\\\addlinespace
permissions & text & YES & & NULL &
\\\addlinespace
activated & tinyint(1) & NO & & 0 &
\\\addlinespace
activation\_code & varchar(255) & YES & MUL & NULL &
\\\addlinespace
activated\_at & timestamp & YES & & NULL &
\\\addlinespace
last\_login & timestamp & YES & & NULL &
\\\addlinespace
persist\_code & varchar(255) & YES & & NULL &
\\\addlinespace
reset\_password\_code & varchar(255) & YES & MUL & NULL &
\\\addlinespace
created\_at & timestamp & NO & & 0000-00-00 00:00:00 &
\\\addlinespace
updated\_at & timestamp & NO & & 0000-00-00 00:00:00 &
\\\addlinespace
\bottomrule
\end{longtable}

ID

username *用户名

password *密码

\paragraph{角色role}\label{ux89d2ux8272role}

admin 管理员,相当于超级管理员状态,拥有除审核外的所有权限

checker 审核员,拥有普通用户和审核的权限

user 普通用户,拥有申请物资,浏览物资等基本功能

\subsubsection{物资信息}\label{ux7269ux8d44ux4fe1ux606f}

\paragraph{material物资表}\label{materialux7269ux8d44ux8868}

ID

name 物资名称

type 物资分类

sum\_n 物资数量(总数)

borrow\_n 物资借出数量

create\_time 物资生成时间

update\_time 物资更新的时间

description 对物资的描述(备注)

status (留用)

comment (留用)

\subsubsection{申请/审核表}\label{ux7533ux8bf7ux5ba1ux6838ux8868}

\paragraph{application表}\label{applicationux8868}

ID

user\_id 用户id

checker\_id 审核者id

reason 申请原因

response 审核结果

request\_time 申请时间

response\_time 审核时间(即为审核员回应的时间)

borrow\_time 借出时间(或预计借出时间)

return\_time 归还时间(预计归还时间)

status 申请状态(pass:审核通过; refuse:拒绝通过; waiting:未审核 )

\paragraph{apply\_resouce表}\label{applyux5fresouceux8868}

ID

application\_id 申请的id号

resource\_id 物资的id号

number 数量

comment 备注

\subsubsection{工作室/会议室表}\label{ux5de5ux4f5cux5ba4ux4f1aux8baeux5ba4ux8868}

和物资的表可重用

\subsubsection{通知信息}\label{ux901aux77e5ux4fe1ux606f}

\paragraph{notice表}\label{noticeux8868}

ID

from\_user 来自哪个用户

content 通知内容

push\_time 通知推送的时间

\paragraph{notic\_to\_user表}\label{noticux5ftoux5fuserux8868}

ID

notice\_id 通知(notice)表的id

user\_id 接收方的id

status 消息状态(已读、未读等)

\section{详细设计}\label{ux8be6ux7ec6ux8bbeux8ba1}

\subsection{登录模块}\label{ux767bux5f55ux6a21ux5757-1}

\subsubsection{功能}\label{ux529fux80fd}

本模块主要实现用户的登入和登出功能。

\subsubsection{输入输出}\label{ux8f93ux5165ux8f93ux51fa}

\paragraph{登入}\label{ux767bux5165}

在URL `http://example.com/login.php' 填写以下表单, \texttt{POST} 到
`http://example.com/login.php'

\begin{itemize}
\itemsep1pt\parskip0pt\parsep0pt
\item
  username
\item
  password
\item
  remember\_me
\end{itemize}

验证成功后跳转到 `首页'

\paragraph{登出}\label{ux767bux51fa}

点击 `登出按钮' -\textgreater{} \texttt{js} 弹出确认框,内容是一个
\texttt{delete} 表单 -\textgreater{} 点击 `确认按钮' -\textgreater{}
提交 \texttt{delete} 表单到 `htto://example.com/logout.php'
-\textgreater{} 跳转回'登入页面'

\subsubsection{接口}\label{ux63a5ux53e3}

\begin{Shaded}
\begin{Highlighting}[]
\NormalTok{interface AuthInterface \{}
    \NormalTok{public function getLogin();}
    \NormalTok{public function postLogout();}
    \NormalTok{public function delLogin();}
\NormalTok{\}}
\end{Highlighting}
\end{Shaded}

\subsection{用户模块}\label{ux7528ux6237ux6a21ux5757-1}

\subsubsection{功能}\label{ux529fux80fd-1}

本模块实现添加帐号、删除账号、修改用户信息等功能。

\subsubsection{输入输出}\label{ux8f93ux5165ux8f93ux51fa-1}

\paragraph{添加帐号}\label{ux6dfbux52a0ux5e10ux53f7-1}

管理员在 `http://example.com/user/create' 页面填写以下表单, 然后
\texttt{POST} 到 `http://example.com/user' 完成添加。

\begin{itemize}
\itemsep1pt\parskip0pt\parsep0pt
\item
  username
\item
  nickname
\item
  password
\item
  role
\end{itemize}

\paragraph{更改用户信息}\label{ux66f4ux6539ux7528ux6237ux4fe1ux606f-1}

管理员在 `http://example.com/user/\{id\}/edit' 页面填写以下表单, 然后
\texttt{PUT} 到 `http://example.com/user/\{id\}' 完成修改。

\begin{itemize}
\itemsep1pt\parskip0pt\parsep0pt
\item
  username
\item
  nickname
\item
  password
\item
  role
\end{itemize}

\paragraph{删除帐号}\label{ux5220ux9664ux5e10ux53f7-1}

管理员发 \texttt{DELETE} 申请到 `http://example.com/user/\{id\}',
完成删除。

\subsubsection{接口}\label{ux63a5ux53e3-1}

\begin{Shaded}
\begin{Highlighting}[]
\NormalTok{interface UserInterface \{}
    \NormalTok{public function getUser();}
    \NormalTok{public function getUserCreate();}
    \NormalTok{public function postUserCreate();}
    \NormalTok{public function getUserUpdate();}
    \NormalTok{public function putUserUpdate();}
    \NormalTok{public function delUser();}
\NormalTok{\}}
\end{Highlighting}
\end{Shaded}

\subsection{首页模块}\label{ux9996ux9875ux6a21ux5757}

\subsubsection{功能}\label{ux529fux80fd-2}

本模块的功能是显示首页。

\subsubsection{输入/输出}\label{ux8f93ux5165ux8f93ux51fa-2}

\texttt{GET} `http://example.com' 就是首页

\subsubsection{接口}\label{ux63a5ux53e3-2}

\begin{Shaded}
\begin{Highlighting}[]
\NormalTok{interface HomeInterface \{}
    \NormalTok{public function getIndex();}
\NormalTok{\}}
\end{Highlighting}
\end{Shaded}

\subsection{物资管理模块}\label{ux7269ux8d44ux7ba1ux7406ux6a21ux5757-1}

\subsection{物资申请模块}\label{ux7269ux8d44ux7533ux8bf7ux6a21ux5757}

\subsubsection{功能}\label{ux529fux80fd-3}

本模块实现用户申请物资、查看申请列表、审核员审核申请的功能。

\subsubsection{输入/输出}\label{ux8f93ux5165ux8f93ux51fa-3}

\paragraph{查看申请列表}\label{ux67e5ux770bux7533ux8bf7ux5217ux8868}

\texttt{GET} `http://example.com/application' 是查看申请列表,
普通用户只能看到已通过的申请,审核员和管理员可以看到全部申请。

\paragraph{申请物资}\label{ux7533ux8bf7ux7269ux8d44-1}

首先,用户 \texttt{GET} `http://example.com/application/create'
打开申请页面, 填写以下表单,然后 `POST' 到
`http://example.com/application' 完成申请。

\begin{itemize}
\itemsep1pt\parskip0pt\parsep0pt
\item
  物资
\item
  申请时间
\item
  归还时间
\item
  申请理由
\item
  用户信息
\end{itemize}

\paragraph{通过/拒绝申请}\label{ux901aux8fc7ux62d2ux7eddux7533ux8bf7}

\texttt{PUT} 通过/拒绝理由到 `http://example/application/\{id\}'
完成操作。

\subsubsection{接口}\label{ux63a5ux53e3-3}

\begin{Shaded}
\begin{Highlighting}[]
\NormalTok{interface ApplicationInterface \{}
    \NormalTok{public function getApplicationCreate();}
    \NormalTok{public function postApplication();}
    \NormalTok{public function getApplicationUpdate();}
    \NormalTok{public function putApplicationUpdate();}
    \NormalTok{public function delApplication();}
\NormalTok{\}}
\end{Highlighting}
\end{Shaded}

\end{document}
